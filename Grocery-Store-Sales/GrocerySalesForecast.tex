% Options for packages loaded elsewhere
\PassOptionsToPackage{unicode}{hyperref}
\PassOptionsToPackage{hyphens}{url}
%
\documentclass[
]{article}
\usepackage{amsmath,amssymb}
\usepackage{iftex}
\ifPDFTeX
  \usepackage[T1]{fontenc}
  \usepackage[utf8]{inputenc}
  \usepackage{textcomp} % provide euro and other symbols
\else % if luatex or xetex
  \usepackage{unicode-math} % this also loads fontspec
  \defaultfontfeatures{Scale=MatchLowercase}
  \defaultfontfeatures[\rmfamily]{Ligatures=TeX,Scale=1}
\fi
\usepackage{lmodern}
\ifPDFTeX\else
  % xetex/luatex font selection
\fi
% Use upquote if available, for straight quotes in verbatim environments
\IfFileExists{upquote.sty}{\usepackage{upquote}}{}
\IfFileExists{microtype.sty}{% use microtype if available
  \usepackage[]{microtype}
  \UseMicrotypeSet[protrusion]{basicmath} % disable protrusion for tt fonts
}{}
\makeatletter
\@ifundefined{KOMAClassName}{% if non-KOMA class
  \IfFileExists{parskip.sty}{%
    \usepackage{parskip}
  }{% else
    \setlength{\parindent}{0pt}
    \setlength{\parskip}{6pt plus 2pt minus 1pt}}
}{% if KOMA class
  \KOMAoptions{parskip=half}}
\makeatother
\usepackage{xcolor}
\usepackage[margin=1in]{geometry}
\usepackage{color}
\usepackage{fancyvrb}
\newcommand{\VerbBar}{|}
\newcommand{\VERB}{\Verb[commandchars=\\\{\}]}
\DefineVerbatimEnvironment{Highlighting}{Verbatim}{commandchars=\\\{\}}
% Add ',fontsize=\small' for more characters per line
\usepackage{framed}
\definecolor{shadecolor}{RGB}{248,248,248}
\newenvironment{Shaded}{\begin{snugshade}}{\end{snugshade}}
\newcommand{\AlertTok}[1]{\textcolor[rgb]{0.94,0.16,0.16}{#1}}
\newcommand{\AnnotationTok}[1]{\textcolor[rgb]{0.56,0.35,0.01}{\textbf{\textit{#1}}}}
\newcommand{\AttributeTok}[1]{\textcolor[rgb]{0.13,0.29,0.53}{#1}}
\newcommand{\BaseNTok}[1]{\textcolor[rgb]{0.00,0.00,0.81}{#1}}
\newcommand{\BuiltInTok}[1]{#1}
\newcommand{\CharTok}[1]{\textcolor[rgb]{0.31,0.60,0.02}{#1}}
\newcommand{\CommentTok}[1]{\textcolor[rgb]{0.56,0.35,0.01}{\textit{#1}}}
\newcommand{\CommentVarTok}[1]{\textcolor[rgb]{0.56,0.35,0.01}{\textbf{\textit{#1}}}}
\newcommand{\ConstantTok}[1]{\textcolor[rgb]{0.56,0.35,0.01}{#1}}
\newcommand{\ControlFlowTok}[1]{\textcolor[rgb]{0.13,0.29,0.53}{\textbf{#1}}}
\newcommand{\DataTypeTok}[1]{\textcolor[rgb]{0.13,0.29,0.53}{#1}}
\newcommand{\DecValTok}[1]{\textcolor[rgb]{0.00,0.00,0.81}{#1}}
\newcommand{\DocumentationTok}[1]{\textcolor[rgb]{0.56,0.35,0.01}{\textbf{\textit{#1}}}}
\newcommand{\ErrorTok}[1]{\textcolor[rgb]{0.64,0.00,0.00}{\textbf{#1}}}
\newcommand{\ExtensionTok}[1]{#1}
\newcommand{\FloatTok}[1]{\textcolor[rgb]{0.00,0.00,0.81}{#1}}
\newcommand{\FunctionTok}[1]{\textcolor[rgb]{0.13,0.29,0.53}{\textbf{#1}}}
\newcommand{\ImportTok}[1]{#1}
\newcommand{\InformationTok}[1]{\textcolor[rgb]{0.56,0.35,0.01}{\textbf{\textit{#1}}}}
\newcommand{\KeywordTok}[1]{\textcolor[rgb]{0.13,0.29,0.53}{\textbf{#1}}}
\newcommand{\NormalTok}[1]{#1}
\newcommand{\OperatorTok}[1]{\textcolor[rgb]{0.81,0.36,0.00}{\textbf{#1}}}
\newcommand{\OtherTok}[1]{\textcolor[rgb]{0.56,0.35,0.01}{#1}}
\newcommand{\PreprocessorTok}[1]{\textcolor[rgb]{0.56,0.35,0.01}{\textit{#1}}}
\newcommand{\RegionMarkerTok}[1]{#1}
\newcommand{\SpecialCharTok}[1]{\textcolor[rgb]{0.81,0.36,0.00}{\textbf{#1}}}
\newcommand{\SpecialStringTok}[1]{\textcolor[rgb]{0.31,0.60,0.02}{#1}}
\newcommand{\StringTok}[1]{\textcolor[rgb]{0.31,0.60,0.02}{#1}}
\newcommand{\VariableTok}[1]{\textcolor[rgb]{0.00,0.00,0.00}{#1}}
\newcommand{\VerbatimStringTok}[1]{\textcolor[rgb]{0.31,0.60,0.02}{#1}}
\newcommand{\WarningTok}[1]{\textcolor[rgb]{0.56,0.35,0.01}{\textbf{\textit{#1}}}}
\usepackage{graphicx}
\makeatletter
\def\maxwidth{\ifdim\Gin@nat@width>\linewidth\linewidth\else\Gin@nat@width\fi}
\def\maxheight{\ifdim\Gin@nat@height>\textheight\textheight\else\Gin@nat@height\fi}
\makeatother
% Scale images if necessary, so that they will not overflow the page
% margins by default, and it is still possible to overwrite the defaults
% using explicit options in \includegraphics[width, height, ...]{}
\setkeys{Gin}{width=\maxwidth,height=\maxheight,keepaspectratio}
% Set default figure placement to htbp
\makeatletter
\def\fps@figure{htbp}
\makeatother
\setlength{\emergencystretch}{3em} % prevent overfull lines
\providecommand{\tightlist}{%
  \setlength{\itemsep}{0pt}\setlength{\parskip}{0pt}}
\setcounter{secnumdepth}{-\maxdimen} % remove section numbering
\ifLuaTeX
  \usepackage{selnolig}  % disable illegal ligatures
\fi
\usepackage{bookmark}
\IfFileExists{xurl.sty}{\usepackage{xurl}}{} % add URL line breaks if available
\urlstyle{same}
\hypersetup{
  hidelinks,
  pdfcreator={LaTeX via pandoc}}

\author{}
\date{\vspace{-2.5em}}

\begin{document}

\begin{center}\rule{0.5\linewidth}{0.5pt}\end{center}

\subsection{output: pdf\_document}\label{output-pdf_document}

\subsection{Read files}\label{read-files}

During the years 1992 to 2021 the Business Cycle Dating Committee of the
National Bureau of Economic Research defined three periods of economic
contraction, 2001(4) to 2001(11), 2008(1) to 2009(6), and 2020(3) to
2020(4).

\textbf{Make separate time series plots for (i) Sales and (ii) log
Sales, and mark the contraction periods on the plots.}

\begin{Shaded}
\begin{Highlighting}[]
\DocumentationTok{\#\# Example data}
\FunctionTok{set.seed}\NormalTok{(}\DecValTok{0}\NormalTok{)}
\NormalTok{dat }\OtherTok{\textless{}{-}}\NormalTok{ sales}

\DocumentationTok{\#\# Determine highlighted regions}
\NormalTok{v }\OtherTok{\textless{}{-}} \FunctionTok{rep}\NormalTok{(}\DecValTok{0}\NormalTok{, }\FunctionTok{max}\NormalTok{(dat}\SpecialCharTok{$}\NormalTok{Time))}
\NormalTok{v[}\FunctionTok{c}\NormalTok{(}\DecValTok{112}\SpecialCharTok{:}\DecValTok{119}\NormalTok{, }\DecValTok{193}\SpecialCharTok{:}\DecValTok{210}\NormalTok{, }\DecValTok{339}\SpecialCharTok{:}\DecValTok{340}\NormalTok{)] }\OtherTok{\textless{}{-}} \DecValTok{1}

\DocumentationTok{\#\# Get the start and end points for highlighted regions}
\NormalTok{inds }\OtherTok{\textless{}{-}} \FunctionTok{diff}\NormalTok{(}\FunctionTok{c}\NormalTok{(}\DecValTok{0}\NormalTok{, v))}
\NormalTok{start }\OtherTok{\textless{}{-}}\NormalTok{ dat}\SpecialCharTok{$}\NormalTok{Time[inds }\SpecialCharTok{==} \DecValTok{1}\NormalTok{]}
\NormalTok{end }\OtherTok{\textless{}{-}}\NormalTok{ dat}\SpecialCharTok{$}\NormalTok{Time[inds }\SpecialCharTok{==} \SpecialCharTok{{-}}\DecValTok{1}\NormalTok{]}
\ControlFlowTok{if}\NormalTok{ (}\FunctionTok{length}\NormalTok{(start) }\SpecialCharTok{\textgreater{}} \FunctionTok{length}\NormalTok{(end)) end }\OtherTok{\textless{}{-}} \FunctionTok{c}\NormalTok{(end, }\FunctionTok{tail}\NormalTok{(dat}\SpecialCharTok{$}\NormalTok{Time, }\DecValTok{1}\NormalTok{))}

\DocumentationTok{\#\# highlight region data}
\NormalTok{rects }\OtherTok{\textless{}{-}} \FunctionTok{data.frame}\NormalTok{(}\AttributeTok{start=}\NormalTok{start, }\AttributeTok{end=}\NormalTok{end, }\AttributeTok{group=}\FunctionTok{seq\_along}\NormalTok{(start))}

\FunctionTok{ggplot}\NormalTok{(}\AttributeTok{data=}\NormalTok{dat, }\FunctionTok{aes}\NormalTok{(Time, Sales)) }\SpecialCharTok{+}
\FunctionTok{theme\_minimal}\NormalTok{() }\SpecialCharTok{+}
\FunctionTok{geom\_line}\NormalTok{(}\AttributeTok{color =} \StringTok{"\#00AFBB"}\NormalTok{, }\AttributeTok{size =}\NormalTok{ .}\DecValTok{6}\NormalTok{) }\SpecialCharTok{+}
\FunctionTok{geom\_rect}\NormalTok{(}\AttributeTok{data=}\NormalTok{rects, }\AttributeTok{inherit.aes=}\ConstantTok{FALSE}\NormalTok{, }\FunctionTok{aes}\NormalTok{(}\AttributeTok{xmin=}\NormalTok{start, }\AttributeTok{xmax=}\NormalTok{end, }\AttributeTok{ymin=}\FunctionTok{min}\NormalTok{(dat}\SpecialCharTok{$}\NormalTok{Sales),}
\AttributeTok{ymax=}\FunctionTok{max}\NormalTok{(dat}\SpecialCharTok{$}\NormalTok{Sales), }\AttributeTok{group=}\NormalTok{group), }\AttributeTok{color=}\StringTok{"transparent"}\NormalTok{, }\AttributeTok{fill=}\StringTok{"orange"}\NormalTok{, }
\AttributeTok{alpha=}\FloatTok{0.3}\NormalTok{) }\SpecialCharTok{+}
\FunctionTok{labs}\NormalTok{(}\AttributeTok{title =} \StringTok{"Grocery Store Sales"}\NormalTok{, }\AttributeTok{subtitle =} \StringTok{"(Contraction period highlighted in Orange)"}\NormalTok{) }
\end{Highlighting}
\end{Shaded}

\begin{verbatim}
## Warning: Using `size` aesthetic for lines was deprecated in ggplot2 3.4.0.
## i Please use `linewidth` instead.
## This warning is displayed once every 8 hours.
## Call `lifecycle::last_lifecycle_warnings()` to see where this warning was
## generated.
\end{verbatim}

\includegraphics{GrocerySalesForecast_files/figure-latex/unnamed-chunk-2-1.pdf}

\begin{Shaded}
\begin{Highlighting}[]
\FunctionTok{ggplot}\NormalTok{(}\AttributeTok{data=}\NormalTok{dat, }\FunctionTok{aes}\NormalTok{(Time, logSales)) }\SpecialCharTok{+}
\FunctionTok{theme\_minimal}\NormalTok{() }\SpecialCharTok{+}
\FunctionTok{geom\_line}\NormalTok{(}\AttributeTok{color =} \StringTok{"\#FC4E07"}\NormalTok{, }\AttributeTok{size =}\NormalTok{ .}\DecValTok{6}\NormalTok{) }\SpecialCharTok{+}
\FunctionTok{geom\_rect}\NormalTok{(}\AttributeTok{data=}\NormalTok{rects, }\AttributeTok{inherit.aes=}\ConstantTok{FALSE}\NormalTok{, }\FunctionTok{aes}\NormalTok{(}\AttributeTok{xmin=}\NormalTok{start, }\AttributeTok{xmax=}\NormalTok{end, }\AttributeTok{ymin=}\FunctionTok{min}\NormalTok{(dat}\SpecialCharTok{$}\NormalTok{logSales),}
\AttributeTok{ymax=}\FunctionTok{max}\NormalTok{(dat}\SpecialCharTok{$}\NormalTok{logSales), }\AttributeTok{group=}\NormalTok{group), }\AttributeTok{color=}\StringTok{"transparent"}\NormalTok{, }\AttributeTok{fill=}\StringTok{"orange"}\NormalTok{, }\AttributeTok{alpha=}\FloatTok{0.3}\NormalTok{) }\SpecialCharTok{+}
\FunctionTok{labs}\NormalTok{(}\AttributeTok{title =} \StringTok{"Log Sales"}\NormalTok{, }\AttributeTok{subtitle =} \StringTok{"(Contraction period highlighted in Orange)"}\NormalTok{) }
\end{Highlighting}
\end{Shaded}

\includegraphics{GrocerySalesForecast_files/figure-latex/unnamed-chunk-3-1.pdf}

\textbf{Analysis: } Upward trend is visible. In the Sales plot (blue
line) Sales seem to increase exponentially or by a constant
multiplicative factor, whereas in the log Sales plot (red line) log
Sales seem to increase by a constant additive amount.

There is annual seasonal component. Grocery sales tend to peak in
December and drop sharply in January and rises to a small peak in the
summer months before peaking again in December. The cycle seems to
repeat annually.

Volatility seems to increase with the number of Sales, and economic
contraction does not seem to affect grocery Sales. There is an outlier
in March 2020 (observation 339), where grocery sales skyrocket due to
the start of the COVID-19 pandemic.

\textbf{Causal Inference: } The most obvious underlying cause of the
peak in December Sales is Christmas holiday shopping - consumers buy
more groceries to do Christmas cooking. The outlying sharp sales peak in
March 2020 is due to the start of COVID-19 pandemic which led Americans
to stockpile groceries.

The multiplicative decomposition model should be fit to the model as we
do see evidence of increasing volatility as Sales go up.

\textbf{Comments on trend and seasonal components: }

The trend and seasonal components are statistically significant. The
partial F-test for both components yield very small p-values which means
we reject the null hypothesis that their regression coefficents are
simultaneously zero (see below for F-test and p-values).

The outlier variable and calendar variables are also statistically
significant. We can see through the summary function that the outlier
variable (obs339) also has a very small p-value. For calendar vaiables,
the partial F-test also yield a very small value which means that
calendar pairs are statistically significant, so we retain them in our
analysis.

\begin{Shaded}
\begin{Highlighting}[]
\CommentTok{\# add dummy variable for outlier, observation at t = 339}
\NormalTok{obs339}\OtherTok{\textless{}{-}}\FunctionTok{rep}\NormalTok{(}\DecValTok{0}\NormalTok{, }\FunctionTok{max}\NormalTok{(sales}\SpecialCharTok{$}\NormalTok{Time))}
\NormalTok{obs339[}\DecValTok{339}\NormalTok{] }\OtherTok{\textless{}{-}} \DecValTok{1}
\NormalTok{sales}\OtherTok{\textless{}{-}}\FunctionTok{data.frame}\NormalTok{(sales,obs339)}

\NormalTok{model1}\OtherTok{\textless{}{-}}\FunctionTok{lm}\NormalTok{(logSales}\SpecialCharTok{\textasciitilde{}}\NormalTok{Time}\SpecialCharTok{+}\FunctionTok{I}\NormalTok{(Time}\SpecialCharTok{\^{}}\DecValTok{2}\NormalTok{)}\SpecialCharTok{+}\FunctionTok{I}\NormalTok{(Time}\SpecialCharTok{\^{}}\DecValTok{3}\NormalTok{)}\SpecialCharTok{+}\FunctionTok{I}\NormalTok{(Time}\SpecialCharTok{\^{}}\DecValTok{4}\NormalTok{)}\SpecialCharTok{+}\NormalTok{fMonth}\SpecialCharTok{+}\NormalTok{obs339}\SpecialCharTok{+}\NormalTok{c348}\SpecialCharTok{+}\NormalTok{s348}\SpecialCharTok{+}\NormalTok{c432}\SpecialCharTok{+}\NormalTok{s432, }\AttributeTok{data =}\NormalTok{ sales);}
\FunctionTok{summary}\NormalTok{(model1)}
\end{Highlighting}
\end{Shaded}

\begin{verbatim}
## 
## Call:
## lm(formula = logSales ~ Time + I(Time^2) + I(Time^3) + I(Time^4) + 
##     fMonth + obs339 + c348 + s348 + c432 + s432, data = sales)
## 
## Residuals:
##       Min        1Q    Median        3Q       Max 
## -0.051396 -0.010798  0.000094  0.010042  0.061814 
## 
## Coefficients:
##               Estimate Std. Error  t value Pr(>|t|)    
## (Intercept)  1.023e+01  5.499e-03 1860.380  < 2e-16 ***
## Time         6.266e-04  1.778e-04    3.524 0.000483 ***
## I(Time^2)    1.654e-05  1.998e-06    8.276 2.96e-15 ***
## I(Time^3)   -6.952e-08  8.311e-09   -8.365 1.59e-15 ***
## I(Time^4)    1.044e-10  1.142e-11    9.141  < 2e-16 ***
## fMonth2     -7.120e-02  4.478e-03  -15.900  < 2e-16 ***
## fMonth3      1.591e-02  4.513e-03    3.526 0.000480 ***
## fMonth4     -7.582e-03  4.476e-03   -1.694 0.091186 .  
## fMonth5      4.729e-02  4.477e-03   10.562  < 2e-16 ***
## fMonth6      1.509e-02  4.476e-03    3.371 0.000836 ***
## fMonth7      4.327e-02  4.478e-03    9.663  < 2e-16 ***
## fMonth8      3.033e-02  4.477e-03    6.776 5.49e-11 ***
## fMonth9     -1.025e-02  4.479e-03   -2.288 0.022730 *  
## fMonth10     1.032e-02  4.477e-03    2.306 0.021697 *  
## fMonth11     1.114e-02  4.480e-03    2.486 0.013382 *  
## fMonth12     7.305e-02  4.479e-03   16.309  < 2e-16 ***
## obs339       1.892e-01  1.787e-02   10.586  < 2e-16 ***
## c348        -9.583e-03  1.298e-03   -7.384 1.20e-12 ***
## s348         5.489e-05  1.292e-03    0.042 0.966133    
## c432        -7.943e-04  1.298e-03   -0.612 0.540962    
## s432         3.425e-03  1.293e-03    2.648 0.008470 ** 
## ---
## Signif. codes:  0 '***' 0.001 '**' 0.01 '*' 0.05 '.' 0.1 ' ' 1
## 
## Residual standard error: 0.01733 on 339 degrees of freedom
## Multiple R-squared:  0.9956, Adjusted R-squared:  0.9953 
## F-statistic:  3800 on 20 and 339 DF,  p-value: < 2.2e-16
\end{verbatim}

Partial F-test to determine the significance of fMonth dummy variables

\begin{Shaded}
\begin{Highlighting}[]
\CommentTok{\# excluding the time polynomials}
\NormalTok{model2}\OtherTok{\textless{}{-}}\FunctionTok{lm}\NormalTok{(logSales}\SpecialCharTok{\textasciitilde{}}\NormalTok{fMonth}\SpecialCharTok{+}\NormalTok{obs339}\SpecialCharTok{+}\NormalTok{c348}\SpecialCharTok{+}\NormalTok{s348}\SpecialCharTok{+}\NormalTok{c432}\SpecialCharTok{+}\NormalTok{s432, }\AttributeTok{data =}\NormalTok{ sales)}

\FunctionTok{anova}\NormalTok{(model2, model1)}
\end{Highlighting}
\end{Shaded}

\begin{verbatim}
## Analysis of Variance Table
## 
## Model 1: logSales ~ fMonth + obs339 + c348 + s348 + c432 + s432
## Model 2: logSales ~ Time + I(Time^2) + I(Time^3) + I(Time^4) + fMonth + 
##     obs339 + c348 + s348 + c432 + s432
##   Res.Df     RSS Df Sum of Sq     F    Pr(>F)    
## 1    343 21.9642                                 
## 2    339  0.1018  4    21.862 18205 < 2.2e-16 ***
## ---
## Signif. codes:  0 '***' 0.001 '**' 0.01 '*' 0.05 '.' 0.1 ' ' 1
\end{verbatim}

Partial F-test to determine the significance of fMonth dummy variables

\begin{Shaded}
\begin{Highlighting}[]
\CommentTok{\# excluding the fMonth dummy variable}
\NormalTok{model3}\OtherTok{\textless{}{-}}\FunctionTok{lm}\NormalTok{(logSales}\SpecialCharTok{\textasciitilde{}}\NormalTok{Time}\SpecialCharTok{+}\FunctionTok{I}\NormalTok{(Time}\SpecialCharTok{\^{}}\DecValTok{2}\NormalTok{)}\SpecialCharTok{+}\FunctionTok{I}\NormalTok{(Time}\SpecialCharTok{\^{}}\DecValTok{3}\NormalTok{)}\SpecialCharTok{+}\FunctionTok{I}\NormalTok{(Time}\SpecialCharTok{\^{}}\DecValTok{4}\NormalTok{)}\SpecialCharTok{+}\NormalTok{obs339}\SpecialCharTok{+}\NormalTok{c348}\SpecialCharTok{+}\NormalTok{s348}\SpecialCharTok{+}
\NormalTok{c432}\SpecialCharTok{+}\NormalTok{s432, }\AttributeTok{data =}\NormalTok{ sales)}

\FunctionTok{anova}\NormalTok{(model3, model1)}
\end{Highlighting}
\end{Shaded}

\begin{verbatim}
## Analysis of Variance Table
## 
## Model 1: logSales ~ Time + I(Time^2) + I(Time^3) + I(Time^4) + obs339 + 
##     c348 + s348 + c432 + s432
## Model 2: logSales ~ Time + I(Time^2) + I(Time^3) + I(Time^4) + fMonth + 
##     obs339 + c348 + s348 + c432 + s432
##   Res.Df     RSS Df Sum of Sq      F    Pr(>F)    
## 1    350 0.52798                                  
## 2    339 0.10177 11    0.4262 129.06 < 2.2e-16 ***
## ---
## Signif. codes:  0 '***' 0.001 '**' 0.01 '*' 0.05 '.' 0.1 ' ' 1
\end{verbatim}

Partial F-test to determine the significance of calendar pair variables

\begin{Shaded}
\begin{Highlighting}[]
\CommentTok{\# excluding the calendar pairs}
\NormalTok{model4}\OtherTok{\textless{}{-}}\FunctionTok{lm}\NormalTok{(logSales}\SpecialCharTok{\textasciitilde{}}\NormalTok{Time}\SpecialCharTok{+}\FunctionTok{I}\NormalTok{(Time}\SpecialCharTok{\^{}}\DecValTok{2}\NormalTok{)}\SpecialCharTok{+}\FunctionTok{I}\NormalTok{(Time}\SpecialCharTok{\^{}}\DecValTok{3}\NormalTok{)}\SpecialCharTok{+}\FunctionTok{I}\NormalTok{(Time}\SpecialCharTok{\^{}}\DecValTok{4}\NormalTok{)}\SpecialCharTok{+}\NormalTok{fMonth}\SpecialCharTok{+}\NormalTok{obs339, }\AttributeTok{data =}\NormalTok{ sales);}

\FunctionTok{anova}\NormalTok{(model4, model1)}
\end{Highlighting}
\end{Shaded}

\begin{verbatim}
## Analysis of Variance Table
## 
## Model 1: logSales ~ Time + I(Time^2) + I(Time^3) + I(Time^4) + fMonth + 
##     obs339
## Model 2: logSales ~ Time + I(Time^2) + I(Time^3) + I(Time^4) + fMonth + 
##     obs339 + c348 + s348 + c432 + s432
##   Res.Df     RSS Df Sum of Sq      F    Pr(>F)    
## 1    343 0.12036                                  
## 2    339 0.10177  4  0.018583 15.475 1.226e-11 ***
## ---
## Signif. codes:  0 '***' 0.001 '**' 0.01 '*' 0.05 '.' 0.1 ' ' 1
\end{verbatim}

\textbf{Comments: } The output of code below produces log the seasonal
indices, so we would need to exponentiate them to obtain the seasonal
indices. The final indices will give us the percentage change of Sales
from month to month.

For example, looking at final indices, since February has seasonal index
of .918 we can expect sales to fall 8.2 percent below the level of the
trend. In December, we can expect sales to increase 6.12 above the level
of trend as it has an index of 1.0612.

More generally, the plot indicates that annually grocery store sales
increase in December and decrease in January and February before rising
again to reach a small peak in the summer months. The percentage factors
by which Sales increase/decrease are listed below.

\begin{Shaded}
\begin{Highlighting}[]
\NormalTok{b1}\OtherTok{\textless{}{-}}\FunctionTok{coef}\NormalTok{(model1)[}\DecValTok{1}\NormalTok{]}
\NormalTok{b2}\OtherTok{\textless{}{-}}\FunctionTok{coef}\NormalTok{(model1)[}\DecValTok{6}\SpecialCharTok{:}\DecValTok{16}\NormalTok{]}\SpecialCharTok{+}\NormalTok{b1}
\NormalTok{b3}\OtherTok{\textless{}{-}}\FunctionTok{c}\NormalTok{(b1,b2)}
\NormalTok{seas}\OtherTok{\textless{}{-}}\NormalTok{b3}\SpecialCharTok{{-}}\FunctionTok{mean}\NormalTok{(b3)}

\NormalTok{seas.ts}\OtherTok{\textless{}{-}}\FunctionTok{ts}\NormalTok{(}\FunctionTok{exp}\NormalTok{(seas))}

\NormalTok{seas}
\end{Highlighting}
\end{Shaded}

\begin{verbatim}
##  (Intercept)      fMonth2      fMonth3      fMonth4      fMonth5      fMonth6 
## -0.013114856 -0.084313971  0.002799790 -0.020696905  0.034175159  0.001973431 
##      fMonth7      fMonth8      fMonth9     fMonth10     fMonth11     fMonth12 
##  0.030150337  0.017220121 -0.023364598 -0.002790114 -0.001974507  0.059936113
\end{verbatim}

\begin{Shaded}
\begin{Highlighting}[]
\FunctionTok{plot}\NormalTok{(seas.ts,}\AttributeTok{ylab=}\StringTok{"seasonal indices"}\NormalTok{,}\AttributeTok{xlab=}\StringTok{"month"}\NormalTok{)}
\end{Highlighting}
\end{Shaded}

\includegraphics{GrocerySalesForecast_files/figure-latex/unnamed-chunk-8-1.pdf}

\begin{Shaded}
\begin{Highlighting}[]
\CommentTok{\#cbind(seas,exp(seas))}

\NormalTok{month }\OtherTok{\textless{}{-}} \FunctionTok{seq}\NormalTok{(}\DecValTok{12}\NormalTok{)}
\NormalTok{seas\_indices }\OtherTok{\textless{}{-}} \FunctionTok{exp}\NormalTok{(seas)}
\NormalTok{seas\_df }\OtherTok{\textless{}{-}} \FunctionTok{data.frame}\NormalTok{(month, seas, seas\_indices)}
\FunctionTok{print.data.frame}\NormalTok{(}\FunctionTok{tbl\_df}\NormalTok{(seas\_df))}
\end{Highlighting}
\end{Shaded}

\begin{verbatim}
## Warning: `tbl_df()` was deprecated in dplyr 1.0.0.
## i Please use `tibble::as_tibble()` instead.
## Call `lifecycle::last_lifecycle_warnings()` to see where this warning was
## generated.
\end{verbatim}

\begin{verbatim}
##    month         seas seas_indices
## 1      1 -0.013114856    0.9869708
## 2      2 -0.084313971    0.9191426
## 3      3  0.002799790    1.0028037
## 4      4 -0.020696905    0.9795158
## 5      5  0.034175159    1.0347658
## 6      6  0.001973431    1.0019754
## 7      7  0.030150337    1.0306095
## 8      8  0.017220121    1.0173692
## 9      9 -0.023364598    0.9769062
## 10    10 -0.002790114    0.9972138
## 11    11 -0.001974507    0.9980274
## 12    12  0.059936113    1.0617687
\end{verbatim}

\begin{enumerate}
\def\labelenumi{(\alph{enumi})}
\setcounter{enumi}{1}
\tightlist
\item
\end{enumerate}

\textbf{Save the residuals from the fit. }

\begin{Shaded}
\begin{Highlighting}[]
\NormalTok{res1 }\OtherTok{\textless{}{-}} \FunctionTok{resid}\NormalTok{(model1)}
\end{Highlighting}
\end{Shaded}

\textbf{(Form a normal quantile plot of these residuals}

\begin{Shaded}
\begin{Highlighting}[]
\FunctionTok{qqnorm}\NormalTok{(res1)}
\FunctionTok{qqline}\NormalTok{(res1)}
\end{Highlighting}
\end{Shaded}

\includegraphics{GrocerySalesForecast_files/figure-latex/unnamed-chunk-11-1.pdf}

\textbf{test the residuals for normality}

\begin{Shaded}
\begin{Highlighting}[]
\FunctionTok{shapiro.test}\NormalTok{(res1)}
\end{Highlighting}
\end{Shaded}

\begin{verbatim}
## 
##  Shapiro-Wilk normality test
## 
## data:  res1
## W = 0.99292, p-value = 0.08761
\end{verbatim}

\textbf{plot the residuals vs.~time}

\begin{Shaded}
\begin{Highlighting}[]
\NormalTok{resid1 }\OtherTok{\textless{}{-}} \FunctionTok{ts}\NormalTok{(res1,}\AttributeTok{start=}\FunctionTok{c}\NormalTok{(}\DecValTok{1992}\NormalTok{,}\DecValTok{1}\NormalTok{),}\AttributeTok{freq=}\DecValTok{12}\NormalTok{)}
\FunctionTok{plot}\NormalTok{(resid1, }\AttributeTok{xlab=}\StringTok{"time"}\NormalTok{,}\AttributeTok{ylab=}\StringTok{"residuals"}\NormalTok{,}\AttributeTok{main=}\StringTok{"Residuals of Model 1"}\NormalTok{)}
\end{Highlighting}
\end{Shaded}

\includegraphics{GrocerySalesForecast_files/figure-latex/unnamed-chunk-13-1.pdf}

\textbf{and plot their autocorrelations. }

\begin{Shaded}
\begin{Highlighting}[]
\FunctionTok{acf}\NormalTok{(}\FunctionTok{ts}\NormalTok{(res1), }\DecValTok{37}\NormalTok{)}
\end{Highlighting}
\end{Shaded}

\includegraphics{GrocerySalesForecast_files/figure-latex/unnamed-chunk-14-1.pdf}

\textbf{Residual Analysis:}

\begin{itemize}
\item
  QQPLOT: The QQPLOT seems to substantiate the assumption that the
  residuals are normally distributed. Most of the quantiles
  (observations) fall on the qqline. There are patterns of departure
  from normality with a few points at the upper and lower tails of the
  plot but not significant enough to the conclude that the residuals are
  not normally distributed.
\item
  Shapiro-Wilk Normality Test: The p-value is 0.08761, which is not
  small enough to reject the null hypothesis that the residuals are
  normally distributed (given a 95\% confidence level, the p-value has
  to fall below 0.05 to conclude that the residuals are not normal).
\item
  Autocorrelation: The autocorrelation function plot indicates strong
  correlation between the residuals at lags 1-12, which indicates that
  the model does not adequately capture trend.
\item
  Residual Analysis: The residuals show that the model does not track
  autocorrelation structure and the trend structure adequately. The
  residuals have a slight upward trend, which means that our model did
  not capture such upward trend adequately. If a model captures trend,
  the plot of residuals should be relatively flat.
\end{itemize}

The residuals also seem to be volatile, which indicate that there maybe
autocorrelation structure which the model does not capture.

\textbf{More Analysis on the residuals}

The residuals show that the model does not track the trend structure
adequately. The residuals have a slight upward trend, which means that
our model did not capture such upward trend adequately. If a model
captures trend, the plot of residuals should be relatively flat.

\textbf{Plotting and comparing seasonal indices}

The seasonal indices and decompose seasonal indices are nearly
identical. This means that our model captures the seasonality component
well.

\begin{Shaded}
\begin{Highlighting}[]
\NormalTok{sales.ts}\OtherTok{\textless{}{-}}\FunctionTok{ts}\NormalTok{(sales}\SpecialCharTok{$}\NormalTok{Sales,}\AttributeTok{freq=}\DecValTok{12}\NormalTok{)}
\NormalTok{sales.decmpsm}\OtherTok{\textless{}{-}}\FunctionTok{decompose}\NormalTok{(sales.ts,}\AttributeTok{type=}\StringTok{"mult"}\NormalTok{)}
\NormalTok{seasdmult1}\OtherTok{\textless{}{-}}\NormalTok{sales.decmpsm}\SpecialCharTok{$}\NormalTok{seasonal}
\NormalTok{seasdmult}\OtherTok{\textless{}{-}}\NormalTok{seasdmult1[}\DecValTok{1}\SpecialCharTok{:}\DecValTok{12}\NormalTok{]}\SpecialCharTok{/}\FunctionTok{prod}\NormalTok{(seasdmult1[}\DecValTok{1}\SpecialCharTok{:}\DecValTok{12}\NormalTok{])}\SpecialCharTok{\^{}}\NormalTok{(}\DecValTok{1}\SpecialCharTok{/}\DecValTok{12}\NormalTok{)}

\NormalTok{seasdmult}
\end{Highlighting}
\end{Shaded}

\begin{verbatim}
##  [1] 0.9875176 0.9182998 1.0102505 0.9792911 1.0345200 1.0008969 1.0305684
##  [8] 1.0164224 0.9756709 0.9961827 0.9963713 1.0611748
\end{verbatim}

Table comparison of seasonal indices and decompose seasonal indices

\begin{Shaded}
\begin{Highlighting}[]
\NormalTok{month }\OtherTok{\textless{}{-}} \FunctionTok{seq}\NormalTok{(}\DecValTok{12}\NormalTok{)}
\NormalTok{seas\_indices }\OtherTok{\textless{}{-}} \FunctionTok{exp}\NormalTok{(seas)}
\NormalTok{seas\_df }\OtherTok{\textless{}{-}} \FunctionTok{data.frame}\NormalTok{(month, seas\_indices, seasdmult)}
\FunctionTok{print.data.frame}\NormalTok{(seas\_df)}
\end{Highlighting}
\end{Shaded}

\begin{verbatim}
##             month seas_indices seasdmult
## (Intercept)     1    0.9869708 0.9875176
## fMonth2         2    0.9191426 0.9182998
## fMonth3         3    1.0028037 1.0102505
## fMonth4         4    0.9795158 0.9792911
## fMonth5         5    1.0347658 1.0345200
## fMonth6         6    1.0019754 1.0008969
## fMonth7         7    1.0306095 1.0305684
## fMonth8         8    1.0173692 1.0164224
## fMonth9         9    0.9769062 0.9756709
## fMonth10       10    0.9972138 0.9961827
## fMonth11       11    0.9980274 0.9963713
## fMonth12       12    1.0617687 1.0611748
\end{verbatim}

Plot comparison of seasonal indices and decompose seasonal indices

\begin{Shaded}
\begin{Highlighting}[]
\FunctionTok{ggplot}\NormalTok{(seas\_df, }\FunctionTok{aes}\NormalTok{(}\AttributeTok{x =}\NormalTok{ month)) }\SpecialCharTok{+} 
  \FunctionTok{geom\_line}\NormalTok{(}\FunctionTok{aes}\NormalTok{(}\AttributeTok{y =}\NormalTok{ seas\_indices, }\AttributeTok{color =} \StringTok{\textquotesingle{}Seasonal Indices\textquotesingle{}}\NormalTok{), }\AttributeTok{size =}\NormalTok{ .}\DecValTok{8}\NormalTok{) }\SpecialCharTok{+} 
  \FunctionTok{geom\_line}\NormalTok{(}\FunctionTok{aes}\NormalTok{(}\AttributeTok{y =}\NormalTok{ seasdmult, }\AttributeTok{color =} \StringTok{\textquotesingle{}Decompose Seasonal Indices\textquotesingle{}}\NormalTok{), }\AttributeTok{size =}\NormalTok{.}\DecValTok{8}\NormalTok{)}
\end{Highlighting}
\end{Shaded}

\includegraphics{GrocerySalesForecast_files/figure-latex/unnamed-chunk-17-1.pdf}

\textbf{Analysis of lag residuals:} The residuals plot is now flat and
less volatile, which means that the model is now closer to yielding
white noise. The autocorrelation function plot now also indicates that
most lags are between the blue lines.

This indicates that the adjustment of including lag variables takes into
account the autocorrelation structure.

\begin{Shaded}
\begin{Highlighting}[]
\NormalTok{lresid}\OtherTok{\textless{}{-}}\FunctionTok{c}\NormalTok{(}\FunctionTok{rep}\NormalTok{(}\ConstantTok{NA}\NormalTok{,}\DecValTok{360}\NormalTok{))}
\NormalTok{lag1resid}\OtherTok{\textless{}{-}}\NormalTok{lresid}
\NormalTok{lag2resid}\OtherTok{\textless{}{-}}\NormalTok{lresid}
\NormalTok{lag3resid}\OtherTok{\textless{}{-}}\NormalTok{lresid}

\NormalTok{lag1resid[}\DecValTok{2}\NormalTok{]}\OtherTok{\textless{}{-}}\FunctionTok{resid}\NormalTok{(model1)[}\DecValTok{1}\NormalTok{]}
\NormalTok{lag1resid[}\DecValTok{3}\NormalTok{]}\OtherTok{\textless{}{-}}\FunctionTok{resid}\NormalTok{(model1)[}\DecValTok{2}\NormalTok{]}
\NormalTok{lag2resid[}\DecValTok{3}\NormalTok{]}\OtherTok{\textless{}{-}}\FunctionTok{resid}\NormalTok{(model1)[}\DecValTok{1}\NormalTok{]}

\ControlFlowTok{for}\NormalTok{(i }\ControlFlowTok{in} \DecValTok{4}\SpecialCharTok{:}\DecValTok{360}\NormalTok{)\{}
\NormalTok{i1}\OtherTok{\textless{}{-}}\NormalTok{i}\DecValTok{{-}1}
\NormalTok{i2}\OtherTok{\textless{}{-}}\NormalTok{i}\DecValTok{{-}2}
\NormalTok{i3}\OtherTok{\textless{}{-}}\NormalTok{i}\DecValTok{{-}3}
\NormalTok{lag1resid[i]}\OtherTok{\textless{}{-}}\FunctionTok{resid}\NormalTok{(model1)[i1]}
\NormalTok{lag2resid[i]}\OtherTok{\textless{}{-}}\FunctionTok{resid}\NormalTok{(model1)[i2]}
\NormalTok{lag3resid[i]}\OtherTok{\textless{}{-}}\FunctionTok{resid}\NormalTok{(model1)[i3]}
\NormalTok{\}}
\end{Highlighting}
\end{Shaded}

\begin{Shaded}
\begin{Highlighting}[]
\NormalTok{sales}\OtherTok{\textless{}{-}}\FunctionTok{data.frame}\NormalTok{(sales,lag1resid,lag2resid,lag3resid)}
\end{Highlighting}
\end{Shaded}

\begin{Shaded}
\begin{Highlighting}[]
\NormalTok{model5}\OtherTok{\textless{}{-}}\FunctionTok{lm}\NormalTok{(logSales}\SpecialCharTok{\textasciitilde{}}\NormalTok{Time}\SpecialCharTok{+}\FunctionTok{I}\NormalTok{(Time}\SpecialCharTok{\^{}}\DecValTok{2}\NormalTok{)}\SpecialCharTok{+}\FunctionTok{I}\NormalTok{(Time}\SpecialCharTok{\^{}}\DecValTok{3}\NormalTok{)}\SpecialCharTok{+}\FunctionTok{I}\NormalTok{(Time}\SpecialCharTok{\^{}}\DecValTok{4}\NormalTok{)}\SpecialCharTok{+}\NormalTok{fMonth}\SpecialCharTok{+}\NormalTok{obs339}\SpecialCharTok{+}
\NormalTok{             c348}\SpecialCharTok{+}\NormalTok{s348}\SpecialCharTok{+}\NormalTok{c432}\SpecialCharTok{+}\NormalTok{s432}\SpecialCharTok{+}
\NormalTok{             lag1resid}\SpecialCharTok{+}\NormalTok{lag2resid}\SpecialCharTok{+}\NormalTok{lag3resid, }\AttributeTok{data =}\NormalTok{ sales)}

\FunctionTok{summary}\NormalTok{(model5)}
\end{Highlighting}
\end{Shaded}

\begin{verbatim}
## 
## Call:
## lm(formula = logSales ~ Time + I(Time^2) + I(Time^3) + I(Time^4) + 
##     fMonth + obs339 + c348 + s348 + c432 + s432 + lag1resid + 
##     lag2resid + lag3resid, data = sales)
## 
## Residuals:
##       Min        1Q    Median        3Q       Max 
## -0.049957 -0.006856 -0.000232  0.007129  0.070343 
## 
## Coefficients:
##               Estimate Std. Error  t value Pr(>|t|)    
## (Intercept)  1.023e+01  4.519e-03 2264.839  < 2e-16 ***
## Time         5.126e-04  1.433e-04    3.576 0.000401 ***
## I(Time^2)    1.780e-05  1.580e-06   11.264  < 2e-16 ***
## I(Time^3)   -7.474e-08  6.494e-09  -11.508  < 2e-16 ***
## I(Time^4)    1.116e-10  8.856e-12   12.601  < 2e-16 ***
## fMonth2     -7.310e-02  3.392e-03  -21.548  < 2e-16 ***
## fMonth3      1.458e-02  3.424e-03    4.259 2.67e-05 ***
## fMonth4     -9.226e-03  3.364e-03   -2.742 0.006431 ** 
## fMonth5      4.564e-02  3.365e-03   13.560  < 2e-16 ***
## fMonth6      1.345e-02  3.364e-03    3.999 7.84e-05 ***
## fMonth7      4.160e-02  3.365e-03   12.363  < 2e-16 ***
## fMonth8      2.871e-02  3.364e-03    8.534 5.07e-16 ***
## fMonth9     -1.190e-02  3.365e-03   -3.537 0.000461 ***
## fMonth10     8.677e-03  3.364e-03    2.579 0.010336 *  
## fMonth11     9.510e-03  3.365e-03    2.826 0.004997 ** 
## fMonth12     7.138e-02  3.366e-03   21.208  < 2e-16 ***
## obs339       2.092e-01  1.352e-02   15.477  < 2e-16 ***
## c348        -9.617e-03  9.708e-04   -9.905  < 2e-16 ***
## s348         2.286e-04  9.666e-04    0.236 0.813193    
## c432        -8.274e-04  9.704e-04   -0.853 0.394493    
## s432         3.677e-03  9.685e-04    3.797 0.000174 ***
## lag1resid    2.510e-01  5.384e-02    4.663 4.53e-06 ***
## lag2resid    3.736e-01  5.119e-02    7.297 2.17e-12 ***
## lag3resid    1.686e-01  5.442e-02    3.099 0.002107 ** 
## ---
## Signif. codes:  0 '***' 0.001 '**' 0.01 '*' 0.05 '.' 0.1 ' ' 1
## 
## Residual standard error: 0.01291 on 333 degrees of freedom
##   (3 observations deleted due to missingness)
## Multiple R-squared:  0.9975, Adjusted R-squared:  0.9974 
## F-statistic:  5832 on 23 and 333 DF,  p-value: < 2.2e-16
\end{verbatim}

\begin{Shaded}
\begin{Highlighting}[]
\NormalTok{res5 }\OtherTok{\textless{}{-}} \FunctionTok{resid}\NormalTok{(model5)}
\NormalTok{resid5 }\OtherTok{\textless{}{-}} \FunctionTok{ts}\NormalTok{(res5,}\AttributeTok{start=}\FunctionTok{c}\NormalTok{(}\DecValTok{1992}\NormalTok{,}\DecValTok{1}\NormalTok{),}\AttributeTok{freq=}\DecValTok{12}\NormalTok{)}
\FunctionTok{plot}\NormalTok{(resid5, }\AttributeTok{xlab=}\StringTok{"time"}\NormalTok{,}\AttributeTok{ylab=}\StringTok{"residuals"}\NormalTok{,}\AttributeTok{main=}\StringTok{"Residuals of Model 5"}\NormalTok{)}
\end{Highlighting}
\end{Shaded}

\includegraphics{GrocerySalesForecast_files/figure-latex/unnamed-chunk-21-1.pdf}

\begin{Shaded}
\begin{Highlighting}[]
\FunctionTok{acf}\NormalTok{(}\FunctionTok{ts}\NormalTok{(res5), }\DecValTok{37}\NormalTok{)}
\end{Highlighting}
\end{Shaded}

\includegraphics{GrocerySalesForecast_files/figure-latex/unnamed-chunk-22-1.pdf}

Investigating the presence of dynamic seasonality. Analyze the data
spanning the years 1992 through 2007 and calculate the static seasonal
estimates.

\begin{Shaded}
\begin{Highlighting}[]
\CommentTok{\# split the Time}
\NormalTok{sales\_i }\OtherTok{\textless{}{-}}\NormalTok{ sales[sales}\SpecialCharTok{$}\NormalTok{Time }\SpecialCharTok{\textless{}=} \DecValTok{192}\NormalTok{, ]}

\CommentTok{\# fit the model}
\NormalTok{model\_i}\OtherTok{\textless{}{-}}\FunctionTok{lm}\NormalTok{(logSales}\SpecialCharTok{\textasciitilde{}}\NormalTok{Time}\SpecialCharTok{+}\FunctionTok{I}\NormalTok{(Time}\SpecialCharTok{\^{}}\DecValTok{2}\NormalTok{)}\SpecialCharTok{+}\FunctionTok{I}\NormalTok{(Time}\SpecialCharTok{\^{}}\DecValTok{3}\NormalTok{)}\SpecialCharTok{+}\FunctionTok{I}\NormalTok{(Time}\SpecialCharTok{\^{}}\DecValTok{4}\NormalTok{)}\SpecialCharTok{+}\NormalTok{fMonth}\SpecialCharTok{+}\NormalTok{obs339}\SpecialCharTok{+}
\NormalTok{              c348}\SpecialCharTok{+}\NormalTok{s348}\SpecialCharTok{+}\NormalTok{c432}\SpecialCharTok{+}\NormalTok{s432, }\AttributeTok{data =}\NormalTok{ sales\_i)}

\CommentTok{\# find the seasonal indices}
\NormalTok{b1}\OtherTok{\textless{}{-}}\FunctionTok{coef}\NormalTok{(model\_i)[}\DecValTok{1}\NormalTok{]}
\NormalTok{b2}\OtherTok{\textless{}{-}}\FunctionTok{coef}\NormalTok{(model\_i)[}\DecValTok{6}\SpecialCharTok{:}\DecValTok{16}\NormalTok{]}\SpecialCharTok{+}\NormalTok{b1}
\NormalTok{b3}\OtherTok{\textless{}{-}}\FunctionTok{c}\NormalTok{(b1,b2)}
\NormalTok{seas\_i}\OtherTok{\textless{}{-}}\NormalTok{b3}\SpecialCharTok{{-}}\FunctionTok{mean}\NormalTok{(b3)}

\NormalTok{seas\_i}
\end{Highlighting}
\end{Shaded}

\begin{verbatim}
##   (Intercept)       fMonth2       fMonth3       fMonth4       fMonth5 
## -0.0205092240 -0.0869616745  0.0004772471 -0.0223475710  0.0321216761 
##       fMonth6       fMonth7       fMonth8       fMonth9      fMonth10 
##  0.0024670588  0.0348806814  0.0189179547 -0.0222580956 -0.0038903604 
##      fMonth11      fMonth12 
## -0.0042826042  0.0713849116
\end{verbatim}

Repeat for the years 2008 through 2019.

\begin{Shaded}
\begin{Highlighting}[]
\CommentTok{\# split the Time}
\NormalTok{sales\_ii }\OtherTok{\textless{}{-}}\NormalTok{ sales[sales}\SpecialCharTok{$}\NormalTok{Time }\SpecialCharTok{\textgreater{}} \DecValTok{192} \SpecialCharTok{\&}\NormalTok{ sales}\SpecialCharTok{$}\NormalTok{Time }\SpecialCharTok{\textless{}=} \DecValTok{336}\NormalTok{, ]}

\CommentTok{\# fit the model}
\NormalTok{model\_ii}\OtherTok{\textless{}{-}}\FunctionTok{lm}\NormalTok{(logSales}\SpecialCharTok{\textasciitilde{}}\NormalTok{Time}\SpecialCharTok{+}\FunctionTok{I}\NormalTok{(Time}\SpecialCharTok{\^{}}\DecValTok{2}\NormalTok{)}\SpecialCharTok{+}\FunctionTok{I}\NormalTok{(Time}\SpecialCharTok{\^{}}\DecValTok{3}\NormalTok{)}\SpecialCharTok{+}\FunctionTok{I}\NormalTok{(Time}\SpecialCharTok{\^{}}\DecValTok{4}\NormalTok{)}\SpecialCharTok{+}\NormalTok{fMonth}\SpecialCharTok{+}\NormalTok{obs339}\SpecialCharTok{+}\NormalTok{c348}\SpecialCharTok{+}\NormalTok{s348}\SpecialCharTok{+}\NormalTok{c432}\SpecialCharTok{+}\NormalTok{s432, }\AttributeTok{data =}\NormalTok{ sales\_ii);}

\CommentTok{\# find the seasonal indices}
\NormalTok{b1}\OtherTok{\textless{}{-}}\FunctionTok{coef}\NormalTok{(model\_ii)[}\DecValTok{1}\NormalTok{]}
\NormalTok{b2}\OtherTok{\textless{}{-}}\FunctionTok{coef}\NormalTok{(model\_ii)[}\DecValTok{6}\SpecialCharTok{:}\DecValTok{16}\NormalTok{]}\SpecialCharTok{+}\NormalTok{b1}
\NormalTok{b3}\OtherTok{\textless{}{-}}\FunctionTok{c}\NormalTok{(b1,b2)}
\NormalTok{seas\_ii}\OtherTok{\textless{}{-}}\NormalTok{b3}\SpecialCharTok{{-}}\FunctionTok{mean}\NormalTok{(b3)}

\NormalTok{seas\_ii}
\end{Highlighting}
\end{Shaded}

\begin{verbatim}
##   (Intercept)       fMonth2       fMonth3       fMonth4       fMonth5 
## -0.0021797639 -0.0780915610  0.0070070144 -0.0211859072  0.0338868547 
##       fMonth6       fMonth7       fMonth8       fMonth9      fMonth10 
## -0.0009865847  0.0241360169  0.0155023179 -0.0254130253 -0.0011725040 
##      fMonth11      fMonth12 
##  0.0009546502  0.0475424922
\end{verbatim}

\textbf{Analysis on Dynamic Seasonality: } There is no evidence of
dynamic seasonality. If there is dynamic seasonality the seasonal
indices should differ significantly, yet the 1992-2007 indices and
2008-2019 indices are relatively similar to each other. A glance at the
plot below reveals that the seasonality component in the two periods are
largely the same, and therefore seasonality is largely static and not
dynamic.

\begin{enumerate}
\def\labelenumi{(\roman{enumi})}
\tightlist
\item
  Table Comparison
\end{enumerate}

\begin{Shaded}
\begin{Highlighting}[]
\NormalTok{month }\OtherTok{\textless{}{-}} \FunctionTok{seq}\NormalTok{(}\DecValTok{12}\NormalTok{)}
\NormalTok{seas\_ind\_i }\OtherTok{\textless{}{-}} \FunctionTok{exp}\NormalTok{(seas\_i)}
\NormalTok{seas\_ind\_ii }\OtherTok{\textless{}{-}} \FunctionTok{exp}\NormalTok{(seas\_ii)}
\NormalTok{seas\_df1 }\OtherTok{\textless{}{-}} \FunctionTok{data.frame}\NormalTok{(month, seas\_ind\_i, seas\_ind\_ii)}
\FunctionTok{print.data.frame}\NormalTok{(seas\_df1)}
\end{Highlighting}
\end{Shaded}

\begin{verbatim}
##             month seas_ind_i seas_ind_ii
## (Intercept)     1  0.9796997   0.9978226
## fMonth2         2  0.9167122   0.9248797
## fMonth3         3  1.0004774   1.0070316
## fMonth4         4  0.9779003   0.9790369
## fMonth5         5  1.0326431   1.0344676
## fMonth6         6  1.0024701   0.9990139
## fMonth7         7  1.0354961   1.0244296
## fMonth8         8  1.0190980   1.0156231
## fMonth9         9  0.9779878   0.9749072
## fMonth10       10  0.9961172   0.9988282
## fMonth11       11  0.9957266   1.0009551
## fMonth12       12  1.0739945   1.0486908
\end{verbatim}

\begin{enumerate}
\def\labelenumi{(\roman{enumi})}
\setcounter{enumi}{1}
\tightlist
\item
  Plot Comparison
\end{enumerate}

\begin{Shaded}
\begin{Highlighting}[]
\FunctionTok{ggplot}\NormalTok{(seas\_df1, }\FunctionTok{aes}\NormalTok{(}\AttributeTok{x =}\NormalTok{ month)) }\SpecialCharTok{+} 
  \FunctionTok{geom\_line}\NormalTok{(}\FunctionTok{aes}\NormalTok{(}\AttributeTok{y =}\NormalTok{ seas\_ind\_i, }\AttributeTok{color =} \StringTok{\textquotesingle{}1992{-}2007\textquotesingle{}}\NormalTok{), }\AttributeTok{size =}\NormalTok{ .}\DecValTok{8}\NormalTok{) }\SpecialCharTok{+} 
  \FunctionTok{geom\_line}\NormalTok{(}\FunctionTok{aes}\NormalTok{(}\AttributeTok{y =}\NormalTok{ seas\_ind\_ii, }\AttributeTok{color =} \StringTok{\textquotesingle{}2008{-}2019\textquotesingle{}}\NormalTok{), }\AttributeTok{size =}\NormalTok{.}\DecValTok{8}\NormalTok{)}
\end{Highlighting}
\end{Shaded}

\includegraphics{GrocerySalesForecast_files/figure-latex/unnamed-chunk-26-1.pdf}

\textbf{Summary: }

Trend: The fourth-degree polynomial trend component captures the trend
well, as indicated by the relatively flat residuals shown in part 2.
However, we do see a slight upward trend and volatility remain in the
residuals plot, which indicate that there remains some trend component
that is not yet captured.

Seasonality: The month dummy variables in the model capture seasonality
well. We compare our model seasonal indices and decompose seasonal
indices and found they are nearly identical.

The seasonality structure of our time series is static. We compare
indices from the years 1992 to 2007 and 2008 to 2019 and found that they
are nearly identical, and thus we conclude that seasonality is static
and not dynamic.

Normality: QQplot and Shapiro-wilk test indicate that there is not
enough evidence to conclude that the residuals are not normally
distributed.

Autocorrelation: The autocorrelation function graph of model 1 indicated
that there are high correlation in lags 1 to 12, which indicates that
our original model did not adequately capture trend. We introduced three
lag residual variables to take into account autocorrelation component of
the time series, and saw that the introduction led to residual plot
which is flatter and less volatile. This indicates that the new
variables capture autocorrelation well and helped the model take into
account autocorrelation which was not accounted for previously, and this
led to residuals that are closer to white-noise level.

\end{document}
